% author_preface.tex
% 2011/02/28, v3.00 gamma

\chapter*{Preface}


The early work of Richard Lewis in the 1990s set the stage for the work reported in the present book. Rick's research on developing a language processing model within the SOAR  architecture  \citep{lewis:phd}  evolved into a sharper focus on developing process models of dependency completion in sentence comprehension. He initiated the use of the cognitive architecture ACT-R to model proactive and retroactive interference effects \citep{lewis:magical}.  The first major elaboration of these ideas appeared in \cite{LewisVasishth2005} and \cite{LewisVasishthVanDyke2006}. In the late 1990s and early 2000s, both Julie Van Dyke and the first author of the present book were Rick's PhD students. Since then, quite a lot of evidence has accumulated that is consistent with Rick's original insight  that dependency completion time (retrieval time in ACT-R parlance) in sentence processing is affected by similarity-based interference. However, some important counterexamples to this proposal have also emerged, and there are some important empirical details relating to retrieval processes that may not be explainable by the general mechanisms posited within ACT-R \citep{AndersonEtAl2004} or other memory architectures. We discuss several of these counterexamples in detail in the present book. More generally, the present book takes stock of the computational modelling done in this context, and situates the modelling within some (but not all) of the important scientific questions in sentence processing research that are actively under consideration today. We hope that this book will be useful to researchers seeking to build on the work presented here, and to develop the next generation of computational models of sentence processing.
 
