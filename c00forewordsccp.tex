\chapter*{Foreword}

In reading a draft of this remarkable book, I was reminded of the title
that Allen Newell chose for his contribution to a meeting celebrating
the scientific contributions of Herbert Simon: "Putting It All
Together".  Newell was both acknowledging Simon's putting so much
together under bounded rationality, but also looking forward to the
theoretical integration made possible by models of cognitive
architecture.  What Shravan Vasishth and Felix Engelmann have offered us
is perhaps the most comprehensive and integrated attempt yet to put it
all together in sentence processing, in a way that begins to do justice
to its rich, cross-linguistic empirical details.

I'll point out below a few of my favourite contributions to integration
that appear in the book.  But I first want to draw the reader's
attention to another thread running through all of the chapters---and
one that is perhaps even more important than the specific details of the
models, explanations and empirical analysis that is focus of each
chapter.

That thread is a sharp critique of our current practices in empirical
and theoretical psycholinguistics.  Indeed, the first two chapters do
not read like the expected triumphant summary of 20 years of empirical
research confirming effects of similarity based interference and other
predictions of our early sentence processing models. On the contrary, it
is a sobering taking-stock of the empirical record and current
methodological practice through the lens of what we have too slowly come
to understand about what is required to make progress.  And what is
required is quite often much larger amounts of carefully collected data,
rigorous statistical analysis and multiple alternative model testing. In
short, this book and the work it reports is part of the larger movement
throughout the psychological and cognitive sciences that is helping us
to wake up to the reality of just how challenging our science really is.
In the case of psycholinguistics, we take for granted that we can infer
internal cognitive and linguistic structure from movements of the eyes
and hands and tongue and lips or fluctuations in electrical potentials
on the scalp. Why did we think that task would be easier than it in fact
is?

But along with this sobering critique, the book also takes us on a kind
of joy ride---letting us experience the joy of a few real advances and
interesting ideas that help us see a little bit further ahead. My own
favourites include the detailed comparison of different retrieval models
(concluding with a rejection of the specific model in \cite{LewisVasishth2005}), the model-based accounts of individual differences and
pathologies (paralleling a renaissance across the field  ranging from
areas such as computational psychiatry to cognitive aging), and the
beginnings of explicit models of adaptive eye-movements that start to do
justice to the flexible and highly adaptive nature of human language
comprehension.

On a more personal note, it is a unique privilege as a scientist to be
able to look back to a collaboration that started over 20 years ago, and
to see how far the work has come. I cannot believe the good fortune I
had to cross paths with Shravan when he was a prodigious graduate
student in Linguistics and I was a young professor in computer science
at Ohio State. The ideas we explored in the early ACT-R models of
parsing were really an evolution and combination of insights of George
Miller, Noam Chomsky and John Anderson. And by now what is generously
referred to in the book as the Lewis \& Vasishth model is really the
Vasishth \& Colleagues model. But in the end, scientific ideas do not
belong to any of us---they belong to the field, and individuals and
teams are but stewards.

Of course there are many gaps, weaknesses and shortcomings in the pages
that follow. But unlike most scientific books, a great many of them are
documented by the authors themselves! And so I am reminded of another
colourful quote (attributed to Warren McCulloch), one that Allen Newell
enjoyed using when advancing his candidate integrated theories : "Don't
bite my finger, look where I'm pointing."  Vasishth and Engelmann are
pointing the way to a better science of sentence processing, and we'd do
well to take a look in their direction. I especially hope that new
students joining the field will do so, and be inspired to take us down
ambitious and imaginative new paths toward integrated and deeply
explanatory theory.

\noindent
\hfill Richard L. Lewis \newline 10 December 2020, Ann Arbor, Michigan, USA
